Commanders may use these advanced rules to further improve their force.

\begin{itemize}

\item {\bfseries Retrain}: Retrain a pilot/crew to a new unit.
Commanders may retrain a pilot/crew when selling a unit and immediately purchasing a replacement unit of the same type or when exchanging the pilots/crew between two units.

Pay 250,000 C-bills multiplied by the difference in BV skill multiplier between their current skill level and 4/5 to retrain the crew/pilot.
For example, a 3/3 pilot has a BV skill multiplier of 1.44, so it costs 110,000 C-bills to retrain a 3/3 pilot for a new unit.
See \emph{BattleTech: TechManual}, page 315, for the BV skill multiplier table.
Add 250,000 C-bills to the retraining cost for pilots/crew with an SPA.
Each pilot/crew has to be retrained when exchanging the pilots/crew between two units.

The old and new unit must be the same type.
For example, a 'Mech pilot can only be retrained into another 'Mech unit.
A Combat Vehicle crew can only retrain to the same type of Combat Vehicle: ground, VTOL, WiGE, or naval.
See \emph{BattleTech: Total Warfare}, page 192, for discussion of the Combat Vehicle types.

\item {\bfseries Advanced Refit}: Pay double the difference in C-bill cost to refit a unit to a different variant that is only available on a different faction list.
The unit must be available in the current era.
A Locust LCT-1V is a common, widely available variant that costs 1,512,400 C-bills while the Locust LCT-6M costs 4,277,500 C-bills is only available to the Free Worlds League and the Word of Blake during the Civil War era.
A commander outside of the Free Worlds League or Word of Blake may pay 5,530,200 C-bills to convert a LCT-1V to a LCT-6M.
Converting back to a variant on the faction list only costs the standard \emph{Refit} cost.
\emph{Advanced Refit} can also be used to convert a unit a unique or experimental variant, if the force does not already contain a unique or experimental unit.
See page \pageref{subsec:force_construction} for restrictions on the single unique or experimental unit in a force.

\item {\bfseries Capture}: Capture a pilot or crew when their unit is destroyed.
A pilot or crew may eject from their 'Mech or abandon their vehicle.
See \emph{BattleTech: Tactical Operations Advanced Rules}, page 164, for rules on ejection and abandoning units.
This pilot or crew may be recovered by a friendly unit or captured by an enemy unit.
A captured pilot or crew may be ransomed, with terms agreed upon between the two commanders.
Alternatively, a captured pilot or crew may be taken as a bondsman and \emph{retrained} as above.

\item {\bfseries Allegiance}: Declare allegiance for to a single employer, such as a Great House, Clan, or wealthy benefactor.
The force receives logistical support in exchange for a portion each mission's objective payments.
Commanders must complete 5 scenarios before they can renounce their \emph{Allegiance}.

\begin{itemize}

\item Receive only 80\% of the payments for objectives

\item Always receive base pay, unless an objective awarded equipment

\end{itemize}

\item {\bfseries Design Quirks}: Commanders may opt into using \emph{Design Quirks} for their entire force.
If a commander opts into using \emph{Design Quirks}, then they always apply to repair, replacement, salvage, and selling costs for all units.
Both sides must agree to use \emph{Design Quirks} for them to apply in a scenario.

See \emph{BattleTech: BattleMech Manual}, page 82, \emph{BattleTech: Campaign Operations}, page 225, or \href{https://sarna.net}{Sarna.net} for a list of all quirks.
See \href{https://megamek.org}{MegaMekLab} or \href{https://sarna.net}{Sarna.net} to determine which quirks apply to units.

Some quirks require modifications to fit in the Outworlds Wastes rules.

\begin{itemize}

\item Two 'Mechs with \emph{Compact 'Mech} may share a dropship bay.

\item \emph{Easy to Maintain} reduces repair and replacement costs by 10\%.

\item \emph{Good Reputation} increases purchase and salvage costs by 10\%.

\item \emph{Modular Weapons} decreases refit costs by 50\%.

\item \emph{Rugged} has no effect.

\item \emph{Ubiquitous} reduces repair and replacement costs by 10\%.

\item \emph{Bad Reputation} decreases purchase and salvage costs by 10\%.

\item \emph{Difficult to Maintain} increases repair and replacement costs by 10\%.

\item \emph{Non-Standard Parts} increases repair and replacement costs by 10\%.

\end{itemize}

\item {\bfseries Custom Design Quirks}: If commanders have opted into using \emph{Design Quirks}, they may purchase additional quirks to customize their units.
If a commander opts into using \emph{Custom Design Quirks}, then they always apply to repair, replacement, salvage, and selling costs for all units.
Both sides must agree to use \emph{Custom Design Quirks} for them to apply in a scenario.

Pay 10\% of the unit's cost in C-bills per positive quirk point to add a positive quirk.
For each positive quirk, commanders must select negative quirks with a total value equal or higher than the positive quirk's point value.
Increase the repair and replacement costs by 10\% for each positive quirk point purchased.
See \emph{BattleTech: Campaign Operations}, page 255, for a table summarizing which quirks may be applied to which unit types.

The following quirks may be used to customize your units:
