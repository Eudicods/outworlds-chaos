\begin{enumerate}

  \item {\bfseries Reconnaissance}: The map contains 15 buildings, 7 of which contain hidden objectives.
    Place the buildings randomly or each side takes turns placing the buildings.
    The defender secretly rolls to determine which buildings hold the objectives.
    Attacking units may search buildings during the End Phase if they are in the target hex (in base to base contact) and no enemy units are in the target hex.
    The search is successful on a 2D6 roll of 7+ for 'Mechs and ground combat vehicles or 5+ for infantry.
    An attacking unit with an active probe may instead search 1 building within range of its active probe on a 2D6 roll of 6+.
    The attacker earns 1 point for each objective they find and the defender earns 1 point for each objective remaining hidden.

  \item {\bfseries Supply Raid}: 3-7 supply depots are on the map, near the center.
    Each supply depot has 1-3 loads of supplies.
    Any unit with hands or cargo capacity can load supplies from the depot if they are in the same hex as a supply depot (in base to base contact) during the End Phase.
    A 'Mech with hand actuators must declare which hand is holding the supplies.
    For units with cargo capacity, a friendly infantry unit must load or unload the supplies during the End Phase.
    Units involved in loading or unloading supplies cannot make weapon or physical attacks during that turn.
    A 'Mech carrying supplies can't fire arm weapons (does -2 damage).
    The supplies have negligible weight.
    There is no movement penalty and jumping units may still jump while carrying supplies.
    Carrying supplies does not decrease the infantry space in a combat vehicle, but all standard infantry mount and dismount rules still apply.
    'Mechs carrying supplies in their hands may drop the supplies at any point during their movement.
    The supplies are automatically dropped if the carrying 'Mech falls or goes prone.
    Units with cargo capacity must be unloaded by a friendly infantry unit.
    Each side cannot retrieve more supplies from the same supply depot until the current supplies from that depot are scored.
    A unit carrying supplies earns 1 Victory Point for bringing the supplies to their home edge.

  \item {\bfseries Zone Control}: 3, 5, or 7 objectives are distributed on the map.
    The locations of the objectives dramatically changes the gameplay.
    The basic configuration is 3 objective along the center of the map and 1 objective halfway between the center and each home edge.
    A side controls an objective if only their units are in or adjacent to the objective (within 2").
    During the End Phase, the side that controls the most objectives earns 2 Victory Point.
    If each side controls the same number of objectives and controls at least 1 objective, then each side earns 1 Victory Point for that round.

  \item {\bfseries Base Defense}: 7 buildings are on the defenders side of the map.
    Each building is medium with a construction factor of 60 (6) and 1-3 levels high (1"-3"), unless the players agree upon a different configuration.
    The attacker earns 1 Victory Point for each building destroyed and the defender earns 1 Victory Point for each building remaining.

  \item {\bfseries King of the Hill}: A hex in the center of the map contains a building with valuable files.
    The building is medium with a construction factor of 60 (6), unless the players agree upon a different configuration.
    The force earns 1 Victory Point for every turn that they have the only infantry units inside of the building at the end of the turn.
    Commanders may add additional bunkers to the center of the map.

\item {\bfseries Assassination}: A VIP needs to be escorted across the battlefield.
    The defender selects a medium or heavy 'Mech from the Periphery General or Pirates list.
    The VIP is a Gunnery 5/ Piloting 4 (Skill 4) pilot and half of the adjusted BV/PV of the 'Mech counts against the defender's BV/PV limit.
    The VIP's 'Mech must cross the map from the defender's home edge to the attacker's home edge.
    The attacker earns 10 Victory Points if this 'Mech is destroyed or 5 Victory Points if this 'Mech receives crippling damage.
    The defender 10 Victory Points if this 'Mech does not receive crippling damage or 5 Victory Points if this 'Mech is crippled but not destroyed.
    The turn limit for this scenario is based upon the terrain and movement profile of the VIP's 'Mech.

  \item {\bfseries Extraction}: Extract 1-3 hidden infantry teams.
    For each hidden unit, the attackers select a hex within 4 rows (8") of the defenders home edge and more than 4 hexes (8") away from the other edges of the map.
    A unit with at least 1 ton of cargo capacity can pick up the target by being in the same hex as a target during the End Phase.
    The target is not destroyed if the carrying unit is destroyed.
    The attackers earn 1 Victory Point for each team extracted and the defenders earn 1 Victory Point for each extraction prevented.

  \item {\bfseries Recovery}: 4-6 disabled 'Mechs are equally spaced along the map diagonal.
    A unit of equal or higher weight class can drag a target 'Mech.
    A friendly unit must be in the same hex (in base to base contact) as the target 'Mech during the End Phase to start dragging it.
    Units without hand actuators must use 1 turn securing the target 'Mech.
    The dragging unit has a 1/2 reduction in their walking MP and cannot jump.
    A target 'Mech cannot fire weapons in one arm (does -1 damage).
    A unit earns Victory Points based upon the Alpha Strike size category of 'Mech returned to their home map edge.

\end{enumerate}
