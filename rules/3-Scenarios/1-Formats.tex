Outworlds Wastes forces are created and tracked using Total Warfare BV, but scenarios can be played in many formats.
Common formats for the scenarios include

\begin{itemize}

\item {\bfseries BattleTech: Total Warfare}: Scenarios for this format will primarily focus on medium scale combat, with each side controlling approximately one lance with supporting assets.

\item {\bfseries BattleTech: Alpha Strike}: Scenarios for this format will primarily focus on large scale combat, with each side controlling approximately one company with supporting assets.

\item {\bfseries BattleTech: Override}: Scenarios for this format will focus on small scale combat, with each side controlling approximately one or two units.

\end{itemize}

Regardless of the scenario format, force maintenance and improvement costs are always calculated per the \emph{Force Management} rules.
Use the rules for the scenario format to define terms such as \emph{destroyed}, \emph{internal damage}, and \emph{critical damage} for the purposes of calculating repair costs.

Alpha Strike cards for all units are available on the \href{http://www.masterunitlist.info}{Master Unit List}.
To convert a unit skill levels from Total Warfare to Alpha Strike, take the average of the Piloting and Gunnery skills, rounded down.
See \emph{Alpha Strike: Commander's Edition}, page 29 for more details.

BattleTech: Override is a rule system that combines MechWarrior: Destiny and Alpha Strike combat rules while drawing some inspiration from Total Warfare.
BattleTech: Override rules and record sheets can be found on the \href{https://dfawargaming.com}{Death From Above Wargaming} website.
These scenarios may require adding additional skills from MechWarrior: Destiny to your pilot.
The BattleTech: Override rules describe how to convert units from Total Warfare to BattleTech: Override.

League organizers may use additional formats, such as BattleTroops.
The scenario format must define terms such as \emph{destroyed}, \emph{internal damage}, and \emph{critical damage} for the purposes of calculating repair costs.
