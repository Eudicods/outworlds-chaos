In casual play, commanders play narrative scenarios designed by league organizers and casual scenarios between forces.
This is an open ended format that lasts as long as commanders want to play.

At any point, a new commander can join the league or a current commander can replace their force with a new one.
Any new force must follow the Force Construction rules.
A commander may run multiple forces so long as the logistics and finances of each force remain separate.

Any optional or advanced rules should be agreed upon per scenario or for the entire league.
For example, league organizers could ban the \emph{Fire For Effect} optional rule for all league games.
Advanced Force Maintenance and Improvement rules should be consistent across the league; for example, \emph{Advanced Refit} needs to be available to all forces or banned from all forces.
Each side should agree upon any available optional or advanced rules for each scenario they play, such as \emph{Multiple Attack Rolls} or \emph{Special Pilot Abilities}.

After each scenario, the players repair and update their forces per the Force Maintenance and Improvements rules.
Commanders should keep track of the outcomes of all scenarios and changes to their force, as shown on the \ifthenelse{\equal{\outworldsMode}{mode-web}}{sample \hyperref[sec:sample_roster]{force roster} and \hyperref[sec:sample_logistics]{scenario logistics tracker}}{\emph{Sample Force Roster} (see p. \pageref{sec:sample_roster}) and \emph{Sample Scenario Logistics Tracking} (see p. \pageref{sec:sample_logistics})}.

The league organizers may advance the era for the league.
Commanders would then need to use the Changing Eras rules for their force.

Additional restrictions may be enforced by league organizers, such as league play only occurring at a specific location, but any such restrictions must be announced in advance.
