\section{Version 1.1}

This errata applies to version 1.1 of the BattleTech: Outworlds Wastes rules.

\subsection{Force Construction}

Clarify rules about infantry bays and superheavy units.
Update the first bullet point on p. 10 of the full rules and p. 5 of the quickstart rules to state:

\begin{itemize}

\item Commanders have a modified Union class DropShip with 15 configurable bays.
Bays may be empty and can changed to a different configuration.
Bay space for all infantry units is shared across bays.
Your entire force must fit onto your DropShip.
Bay limits are in the table below.

\begin{table}[!h]
\ifthenelse{\not \equal{\outworldsMode}{mode-web}}{\fontfamily{Montserrat-LF}}{\small}\selectfont
\centering
\newcolumntype{R}[1]{>{\raggedleft\let\newline\\\arraybackslash\hspace{0pt}}m{#1}}
\begin{tabular}{!{\Vline{1pt}} m{10em} m{20em} R{3.8em} !{\Vline{1pt}}}
\Hline{1pt}
\rowcolor{black!30}  \bfseries{Bay Type} & \bfseries{Capacity} & \bfseries{Limit} \\
\Hline{1pt}
'Mech            & 1 'Mech or 1/2 superheavy 'Mech    & 12 bays \\
Combat Vehicle   & 2 vehicles or 1 superheavy vehicle & 5 bays  \\
Aerospace        & 1 aerospace unit                   & 2 bays  \\
ProtoMech        & 5 ProtoMechs                       & 2 bays  \\
Infantry         & 15 tons or 1 unit over 15 tons     & 2 bays  \\
\Hline{1pt}
\end{tabular}
\caption*{DropShip Bay Limits}
\end{table}

Support, Advanced Support, and Advanced Aerospace units are not permitted.
Illegal designs and units over 200 tons are also not permitted.
Units over 100 tons, such as superheavy 'Mechs, require double the bay space as standard units.

\end{itemize}

Add the requirement that unit skill levels can differ by no more than 3.
Update the last bullet point on p. 10 of the full rules and p. 5 of the quickstart rules to state:

\begin{itemize}

\item The BV cost of a unit includes the skill level.
The initial skill levels for a unit may be no better than Gunnery 3/Piloting 4 and cannot differ by more than 3.
Average skill levels for factions and units are given on \emph{BattleTech: Total Warfare} p. 40.
ProtoMechs always have Piloting 5 and infantry units without anti-'Mech equipment have Anti-'Mech 5, because these skills are not used for these units.

\end{itemize}

\subsection{Event Force Construction}

Update the fourth and fifth bullet point on p. 4 of the Event rules:

\begin{itemize}

\item Each force must have at least one Battle Armor unit.
If Battle Armor is not available to the faction in the era, then the force must contain at least one non-mechanized conventional infantry unit.

\item Each force must have one unit capable of carrying the Battle Armor (or infantry) unit.
An OmniMech can carry standard Battle Armor.
BattleMechs can carry Battle Armor equipped with magnetic clamps, but receive a -1 MP/-2 inch reduction to their Walking MP (p. 227, \emph{TW}, p. 39, \emph{AS: CE}).

\end{itemize}

Add the requirement that unit skill levels can differ by no more than 3.
Update the last bullet point on p. 4 of the Event rules:

\begin{itemize}

\item The BV costs of a unit includes all adjustments, to include skill levels, C\textsuperscript{3}, and TAG.
Skill levels should generally be close to the average skill levels (see p. 40, \emph{TW}).
A unit may be no better than Gunnery 3/Piloting 4 and no worse than Gunnery 5/Piloting 6.
ProtoMechs always have Piloting 5 and infantry units without anti-'Mech equipment have Anti-'Mech 5.

\end{itemize}

\subsection{Force Maintenance}

The paragraph for Salvage had a grammatical error.
That paragraph on p. 14 of the full rules, p. 7 of the quickstart rules, and p. 5 of the event rules should read:

{\bfseries Salvage}: Recover enemy units that that were \emph{destroyed} in a scenario.
Pay 50\% the C-bill cost, rounded up, to add salvaged enemy units to your force.
A War Crow Prime costs 22,057,358 C-bills, so it costs 11,028,679 C-bills to add a salvaged War Crow Prime to your force.
The new unit starts at skill 4/5 and can be \emph{trained}.
Alternatively, sell the salvaged unit to earn 25\% of the C-bill cost.
A salvaged War Crow Prime could be sold to earn 5,514,340 C-bills.

Add the requirement that unit skill levels can differ by no more than 3.
Update the Train rule on p. 14 of the full rules, p3. 7 of the quickstart rules to say:

{\bfseries Train}: Pay 500,000 C-bills multiplied by the difference in BV skill multiplier to improve a unit's skill levels.
For example, a Gunnery 4/Piloting 5 pilot has a BV skill multiplier of 1.0 and a 3/4 pilot has a BV skill multiplier of 1.32.
Therefore, it costs 160,000 C-bills to train a 4/5 pilot to 3/4.
Units cannot be upgraded past 1/2.
New units and units that did not participate in the most recent scenario cannot be upgraded past 3/4.
Skills cannot differ by more than 3.
See \emph{BattleTech: TechManual} p. 315 for the BV skill multiplier table.
A unit's skill levels may be degraded at no C-bill cost.
ProtoMechs and infantry units without anti-'Mech equipment have Piloting/Anti-'Mech 5.

Update the Train rule on p. 5 of the event rules to say:

{\bfseries Train}: Pay 500,000 C-bills multiplied by the difference in BV skill multiplier to improve a unit's skill levels.
For example, a Gunnery 4/Piloting 5 pilot has a BV skill multiplier of 1.0 and a 3/4 pilot has a BV skill multiplier of 1.32.
Therefore, it costs 160,000 C-bills to train a 4/5 pilot to 3/4.
Units cannot be upgraded past 2/3.
Skills cannot differ by more than 3.
See \emph{BattleTech: TechManual} p. 315 for the BV skill multiplier table.
A unit's skill levels may be degraded at no C-bill cost.
ProtoMechs and infantry units without anti-'Mech equipment have Piloting/Anti-'Mech 5.

\subsection{Advanced Force Maintenance and Improvements}

Add the following two new optional rules on p.16 of the full rules:

\begin{description}

\item {\bfseries Conscripts}: Foot infantry may be treated as conscripts.
These units have no cost to \emph{recruit} or \emph{replace} and cannot be sold as \emph{salvage}.
These units must be \emph{purchased} as normal to be added to a force.
Conscripted troops may be no better than Gunnery 4/Anti-'Mech 5 and cannot have Special Pilot Abilities (SPAs).
Commanders may have elite infantry who can be higher skill and receive SPAs.

\item {\bfseries Anti-'Mech Costs}: Some infantry units, such as Anti-'Mech jump infantry, can have particularly high C-bill costs.
The relatively low BV cost for these units can be exploited to give commanders disproportionate initial C-bills.
League organizers may reduce the Anti-'Mech Training and Equipment cost multiplier (see p.282 TM) from 5.0 to 2.0 to mitigate this issue.
Commanders multiply the C-bill costs of all units with Anti-'Mech training and equipment by 0.4.

\end{description}

\subsection{DropShip Customization - Maintenance}

Replace the first paragraph on p. 21 of the full rules with the following paragraphs:

Improve the ability to repair, replace, or refit units in a maintenance category.
Also, commanders may fix a limited number of through-armor critical hits or motive system damage without paying the \emph{repair} cost.
The maintenance categories are 'Mechs, ground Combat Vehicles, VTOL Combat Vehicles, Aerospace units, and ProtoMechs.

This customization stacks with quirks.
A unit that is \emph{Easy to Maintain} on a DropShip with level 1 in maintenance for that unit type is 15\% cheaper to \emph{repair} or \emph{replace} while a unit that is \emph{Hard to Maintain} would only be 5\% more expensive to \emph{repair} or \emph{replace}.

\subsection{DropShip Customization - Support Bays}

Add the following text at the bottom of p.22 of the full rules:

\emph{\bfseries Support Bays}: A vehicle may be used to convert combat vehicle bay space into DropShip customization levels.
Using a vehicle in this fashion fills the entire bay.
The vehicle must cost at least 500,000 C-bills; if the vehicle costs less, pay the difference to use the vehicle in this way.
Only one level of a type may be added by a vehicle and customization cannot exceed level 3.
The vehicles may be used in scenarios but do not grant a level of customization if \emph{damaged} or \emph{destroyed} unless \emph{repaired} or \emph{replaced}.

\begin{itemize}

\item BattleMech Recovery Vehicles give 1 level of Machine Shop

\item Vehicles with at least 6 tons of MASH equipment give 1 level of Medical Bay

\end{itemize}

\subsection{Casual Scenario Primary Objectives}

The following changes fix Alpha Strike compatibility for Casual Scenarios.
Update the following items on p. 28 of the full rules and p. 9 of the quickstart rules  as follows:

\begin{enumerate}

\item {\bfseries Reconnaissance}: The map contains 15 buildings, 7 of which contain hidden objectives.
Place the buildings randomly or each side takes turns placing the buildings.
The defender secretly rolls to determine which buildings hold the objectives.
Attacking units may search buildings during the End Phase if they are in the target hex (in base to base contact with the building, Alpha Strike) and no enemy units are in the target hex.
The search is successful on a 2D6 roll of 7+ for 'Mechs and ground combat vehicles or 5+ for infantry.
An attacking unit with an active probe may instead search 1 building within range of its active probe on a 2D6 roll of 6+.
The attacker earns 1,000,000 C-bills for each objective they find and the defender earns 1,000,000 C-bills for each objective remaining hidden.

\item {\bfseries Supply Raid}: 3-7 supply depots are on the map, near the center.
Each supply depot has 1-3 loads of supplies.
Any unit with hands or cargo capacity can load supplies from the depot if they in the same hex as a supply depot (in base to base contact, Alpha Strike) during the End Phase.
For a 'Mech with hand actuators, the unit must declare which hand is holding the supplies.
For units with cargo capacity, a friendly infantry unit must load or unload the supplies during the End Phase.
Units involved in loading or unloading supplies cannot make weapon or physical attacks during that turn.
A 'Mech carrying supplies can't fire arm weapons (BattleTech) or does -2 damage (Alpha Strike).
The supplies are light enough that there is no movement penalty; jumping units may still jump while carrying supplies.
'Mechs carrying supplies in their hands may drop the supplies at any point during their movement.
The supplies are automatically dropped if the carrying 'Mech falls or goes prone.
Units with cargo capacity must be unloaded by a friendly infantry unit.
Each side cannot score from the same supply depot twice until they score from every other supply depot.
A unit carrying supplies earns a portion of 7,000,000 C-bills for bringing the supplies to their home edge based upon the total number of loads of supplies available.

\item {\bfseries Zone Control}: 3, 5, or 7 key points are distributed on the map.
The locations of the key points dramatically changes the gameplay.
The basic configuration is 3 key points along the center of the map and 1 key point halfway between the center and each home edge.
A side controls a key point if only their units are in or adjacent to the key point (within 2", Alpha Strike).
During the End Phase, the side that controls the most key points earns 1,000,000 C-bills.
If each side controls the same number of points and controls at least 1 point, then each side earns 500,000 C-bills for that round.

\item {\bfseries Base Defense}: 7 buildings are on the defenders side of the map.
Each building is medium with a construction factor of 60 (6, Alpha Strike) and 1-3 levels high (1"-3", Alpha Strike), unless the players agree upon a different configuration.
The attacker earns 1,000,000 C-bills for each building destroyed and the defender earns 1,000,000 C-bills for each building remaining.

\item {\bfseries King of the Hill}: A hex in the center of the map contains a building with valuable files.
The building is medium with a construction factor of 60 (6, Alpha Strike), unless the players agree upon a different configuration.
The force earns 1,000,000 C-bills for every turn that they have the only infantry units inside of the building at the end of the turn.

\item {\bfseries Assassination}: A local militia commander needs to be escorted across the battlefield.
The defender selects a medium or heavy 'Mech from the Periphery General or Pirates list.
The commander is a Gunnery 5/ Piloting 4 (Skill 4, Alpha Strike) pilot and half of the adjusted BV/PV of the 'Mech counts against the defender's BV/PV limit.
The commander's 'Mech must cross the map from the defender's home edge to the attacker's home edge.
The attacker earns 7,000,000 C-bills if this 'Mech is destroyed or 3,500,000 C-bills if this 'Mech receives crippling damage.
The defender 7,000,000 C-bills if this 'Mech does not receive crippling damage or 3,500,000 C-bills if this 'Mech is crippled but not destroyed.

\item {\bfseries Extraction}: Extract 1-3 hidden infantry teams.
For each hidden unit, the attackers select a hex within 4 rows (8", Alpha Strike) of the defenders home edge and more than 4 hexes (8", Alpha Strike) away from the other edges of the map.
A unit with at least 1 ton of cargo capacity can pick up the target by being in the same hex as a target during the End Phase.
The target is not destroyed if the carrying unit is destroyed.
A unit earns a portion of 7,000,000 C-bills by exiting their home edge while carrying a target.

\item {\bfseries Recovery}: 4-6 disabled 'Mechs are equally spaced along the map diagonal.
A unit of equal or higher weight class can drag a target 'Mech.
A friendly unit must be in the same hex (in base to base contact, Alpha Strike) as the target 'Mech during the End Phase to start dragging it.
Units without hand actuators must use 1 turn securing the target 'Mech.
The dragging unit has a 1/2 reduction in their walking MP and cannot jump.
A target 'Mech cannot fire weapons in one arm (BattleTech) or does -1 damage (Alpha Strike).
A unit earns a proportion of 7,000,000 C-bills for each 'Mech returned to their home map edge.

\end{enumerate}

Add the following sentences to the beginning of p. 29 of the full rules and p. 10 of the quickstart rules:

These objectives may be adapted to best support the scenario format or the narrative the players are creating, as long as all commanders agree.
When playing these objectives in Alpha Strike, 1 hex is equivalent to 2 inches and "in the same hex" means "in base to base contact".
