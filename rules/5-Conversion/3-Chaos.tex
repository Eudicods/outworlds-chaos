\emph{BattleTech: Chaos Campaign} provides campaign rules that have lower complexity than \emph{BattleTech: Campaign Operations} and is available as a free PDF.
The Chaos Campaign rules are used for campaigns in \emph{BattleTech: Chaos Campaign: Succession Wars} as well as \emph{BattleTech: Spotlight On} and \emph{BattleTech: Turning Points} publications and \emph{Shrapnel}, the BattleTech magazine.
The \emph{BattleTech: Mercenaries} box set extends these rules and adds a contract system to \emph{BattleTech: Chaos Campaigns}.

\emph{BattleTech: Outworlds Wastes} can be used as the logistics system for the \emph{BattleTech: Mercenaries} box set extension of the \emph{Chaos Campaign} system.

1,000 Supply Points for the \emph{Mercenary Chaos Campaign} system is worth 2,000,000 C-bills in \emph{BattleTech: Outworlds Wastes}.
A standard 10,000 BV \emph{BattleTech: Outworlds Wastes} force is equivalent to a Scale 2 force in \emph{Mercenary Chaos Campaign}.
These conversion rules assume one scenario per month for the ease of computing logistics costs.

These following key terms in \emph{Mercenary Chaos Campaign} interact with \emph{BattleTech: Outworlds Wastes} as described below.

\begin{itemize}

\item {\bfseries Base Pay}: \emph{BattleTech: Outworlds Wastes} assumes the employer or supporting faction covers 100\% of the base operating costs for the commander.
The base operating cost for an \emph{BattleTech: Outworlds Wastes} unit is 2,000,000 C-bills.
If the contract has lower than 100\% base pay, pay these operating costs after each scenario.
For example, if the contract has 80\% base pay, the commander pays 400,000 C-bills after each scenario as basic operating costs.
Similarly, if the contract has higher than 100\% base pay, each extra C-bills after each scenario.

\item {\bfseries Straight Support}: Reduce the \emph{repair} and \emph{recruit} costs by the percentage of Straight Support in the contract.
For example, if the contract offers 80\% Straight Support, the commander only pays 20\% of the \emph{repair} and \emph{replace} costs.

\item {\bfseries Battle Support}: Reduce the \emph{replace} costs by the percentage of Battle Support in the contract.
For example, if the contract offers 10\% Battle Support, the commander only pays 90\% of the \emph{replace} costs.
Note, a contract with Battle Support also grants 100\% Straight Support, so the commander only pays the remaining portion of the \emph{replace} costs and none of the \emph{repair} or \emph{recruit} costs.

\item {\bfseries Salvage Rights}: Only collect the percent of the revenue from \emph{salvage} granted by the Salvage Rights.
For example, if the contract offers 60\% Salvage Rights, then the commander only earns 15\% of the C-bill cost when selling salvaged units instead of the typical 25\% of the C-bill cost.
Also, the Salvage Rights limit how many units the commander may salvage and add to their force.
Multiply the number of enemy units destroyed by the Salvage Rights percentage and round down to determine how many units may be salvaged and added to their force.
A minimum of 1 unit can always be salvaged and added to their force.
For example, if a commander destroys 5 enemy units and has 60\% Salvage Rights, then they may add up to 3 of those units to their force.
If a commander destroys 2 enemy units and has 40\% Salvage Rights, then they may add only 1 of those units to their force.

\item {\bfseries Command Rights}: The specific scenarios determine what effects, if any, Command Rights have on the scenario.

\item {\bfseries Transportation}: If using Transportation costs, use the 1,000 Supply Points to 2,000,000 C-bills conversion to compute Transportation costs.
The employer or supporting faction covers the negotiated portion of the C-bill cost.

\end{itemize}

The typical contract would grand 100\% Base Pay, 40\% Salvage Rights, 10\% Battle Support (with 100\% Straight Support), House Command Rights and 50\% Transportation.
As described in the \emph{BattleTech: Mercenaries} box set, commanders may negotiate to adjust the terms of their contract.

Organizers may score scenarios by using the Supply Point payments given in the \emph{Mercenary Chaos Campaign} system and using the 1,000 Supply Points to 2,000,000 C-bills conversion.
This scoring only provides three possible payments, 50\%, 100\%, or 150\% of the commander's Base Pay.
For a more granular scoring, take the percentage of the total victory points earned by each side and multiply by 2 to give the percentage of the Base Pay earned.
For example, if one side earned 8 victory points and the other side earned 13 victory points out of a total 25 possible victory points, then award 64\% of their base pay to the first side and 104\% of their base pay to the second side.

If the scenario has secondary objectives, the maximum secondary objective payment should be 1,000,000 C-bills or 50\% of the commander's base pay.

League organizers may make adjustments to these guidelines as desired.
