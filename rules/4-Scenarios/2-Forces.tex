Both sides should agree upon a BV (or Point Value, PV) and unit count limit before starting the scenario.
A typical BV limit would be 6,000 BV per side for 1v1 or 10,000 BV per side for 2v2 with \emph{BattleTech: Total Warfare}.
A typical PV limit would be 150 PV per side for 1v1 or 250 PV per side for 2v2 for \emph{BattleTech: Alpha Strike}.
A typical unit limit depends upon the format but would be approximately 7 units per side for 1v1 or 10 units per side for 2v2.
Additional limits on specific unit types, such as 2 infantry/Battle Armor units per side, can be imposed as well.

Scenario forces should include all applicable adjustments in their BV/PV calculations, to include TAG, C\textsuperscript{3}, and pilot skill adjustments.
See \pageref{sec:force_bv_adjustments} for a summary of the most common adjustments.
See \emph{BattleTech: TechManual}, page 202 and all relevant errata for full details on calculating BV.

Scenarios can be played with higher BV/PV limits, but the C-bills awarded should be adjusted if the limits are more than 25\% above or below the typical limits.
For example, an Alpha Strike 300 PV per side 1v1 scenario would have its C-bill payments doubled compared to the standard Alpha Strike 150 PV per side 1v1 scenario.
A Total Warfare 4,000 BV per side 1v1 scenario would have its C-bills payments scaled by 2/3 compared to the standard Total Warfare 6,000 BV per side 1v1 scenario.

Alpha Strike scenarios may be played with BV limits instead of PV limits.
Commanders would select units to meet the BV limit but use the Alpha Strike cards and rules for the scenario.
